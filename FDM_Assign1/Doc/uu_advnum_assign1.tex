\documentclass[12pt]{article}

% Packages
\usepackage{float}
\usepackage{amsmath}
\usepackage{amsfonts}
\usepackage{subcaption}
\usepackage{graphicx}
\usepackage{listingsutf8}
\usepackage[utf8]{inputenc}
\usepackage{xcolor}
\usepackage[a4paper, total={16cm, 23.7cm}]{geometry}
\usepackage[colorlinks=true,urlcolor=blue]{hyperref}
\usepackage{color}

% Compiler/IDE directives
% !TeX spellcheck = en_US

% General
\fontfamily{phv}
\captionsetup{width=0.8\textwidth}

% Graphicx
\graphicspath{ {./images/} }
 
% Listings
\definecolor{codegreen}{rgb}{0,0.6,0}
\definecolor{codegray}{rgb}{0.5,0.5,0.5}
\definecolor{codepurple}{rgb}{0.58,0,0.82}
\definecolor{backcolour}{rgb}{0.95,0.95,0.92} 
\lstdefinestyle{mystyle}{
    backgroundcolor=\color{backcolour},   
    commentstyle=\color{codegreen},
    keywordstyle=\color{magenta},
    numberstyle=\tiny\color{codegray},
    stringstyle=\color{codepurple},
    basicstyle=\fontsize{7}{9}\selectfont\ttfamily,
    breakatwhitespace=false,         
    breaklines=true,                 
    captionpos=b,                    
    keepspaces=true,                 
    numbers=left,                    
    numbersep=5pt,                  
    showspaces=false,                
    showstringspaces=false,
    showtabs=false,                  
    tabsize=2
}
\lstset{style=mystyle}

% Geometry
\DeclareMathSizes{12}{13}{10}{8}
\setlength\parindent{0.7cm}

% Helper commands
\newcommand*\diff{\mathop{}\!\mathrm{d}}
\newcommand*\Diff[1]{\mathop{}\!\mathrm{d^#1}}
\newcommand\tab[1][.7cm]{\hspace*{#1}}
\renewcommand{\refname}{}
\newcommand{\code}[1]{\texttt{#1}}
\newcommand{\norm}[1]{\left\lVert#1\right\rVert}


%----------------------------------------------------------------------------
% THE DOCUMENT
%----------------------------------------------------------------------------
\begin{document}

%----------------------------------------------------------------------------
% TITLE
%----------------------------------------------------------------------------
\begin{center}
	\Huge Advanced Numerical Methods\\
	\Large Assignment 1\\
	\vspace{1pc}
	\huge Péter Kardos \\
	\large 2019
\end{center}

Just uploading something.

\section{Task 1}

\subsection{Energy method}

The energy method on the continuous problem is used to derive the boundary terms.

\begin{equation}
    \begin{split}
        Cu_t = Au_x\\    
        u = 
        \begin{bmatrix}
            E\\
            H    
        \end{bmatrix}
        A = 
        \begin{bmatrix}
            0 & 1\\
            1 & 0
        \end{bmatrix}
        C = 
        \begin{bmatrix}
            \epsilon & 0\\
            0 & \mu    
        \end{bmatrix}
    \end{split}
\end{equation}

\noindent Multiply by $u^T$ from the left an integrate by parts:

\begin{equation}\label{eq_cont_energy_ibp}
    (u^T, Cu_t) = (u^T, Au_x) = -(u^T_x, Au) + [u^TAu]_{x_l}^{x_r}
\end{equation}

\noindent Calculate the transpose:

\begin{equation}\label{eq_cont_energy_transp}
    ((Cu_t)^T, u) = ((Au_x)^T, u) = (u_x^TA^T, u) = (u_x^TA, u) = (u_x^T, Au)
\end{equation}

\noindent Adding (\ref{eq_cont_energy_ibp}) and (\ref{eq_cont_energy_transp}):

\begin{equation}
    (u^T, Cu_t) + ((Cu_t)^T, u) = \frac{d}{dt}\norm{u}^2 = (u_x^T, Au) - (u^T_x, Au) + [u^TAu]_{x_l}^{x_r} = [u^TAu]_{x_l}^{x_r}
\end{equation}

\noindent We get the boundary terms:

\begin{equation}
    [u^TAu]_{x_l}^{x_r} = u^{(1)}u^{(2)}|_{x_r} - u^{(1)}u^{(2)}|_{x_l}
\end{equation}



\section{Task 2}

\section{Task 3}


\end{document}